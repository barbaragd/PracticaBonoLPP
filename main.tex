\documentclass{article}
\usepackage[utf8]{inputenc}

\title{Búsqueda}
\author{Equipo 8}
\date{3 de diciembre del 2019}

\begin{document}

\tableofcontents

\maketitle

\section{Introducción}

\section{Objetivo}
El objetivo es entender el concepto de Búsqueda. Para ello, se ha querido explicar dicho concepto mediante la técnica de búsqueda siguiente: 'Jump Search'.
Este algoritmo consiste en recorrer un vector ordenado dando 'saltos', es decir, si definimos el salto de 4 posiciones, cada vez que saltamos, comparamos el valor de la posición con el valor objetivo, si es menor, procedemos a volver a saltar, si es mayor, recorremos hacia atrás las 4 posiciones que hemos saltado hasta encontrar el número que estamos buscando. 

\section{Material}
\begin{itemize}
\item Cartulinas tamaño folio, una para cada niño, y del mismo color.
\end{itemize}

\section{Metodología}
Dependiendo del número de niños a realizar la actividad, nos dividiremos en dos, o tres grupos. Como cada niño será una posición del vector ordenado, podría resultar más fácil que el vector no fuera de gran tamaño. 
En cada uno de los grupos, uno de los niños representará el valor a buscar entre los números del vector, se le dará una cartulina con el número a buscar. El resto de los niños, cogerá cada uno una cartulina y se colocarán en orden (de menor a mayor). Podría resultar más interesante si los niños ordenados no le mostrasen al niño que busca la cartulina que tienen con el número. A continuación, el niño que está buscando su número entre los demás, comenzará a desarrollar el algoritmo.
Para ello, comenzará colocándose en frente del primer niño, y a partir de ahí contará hasta 4 hasta el siguiente niño. Cuando llega a este, le pregunta si es el número que está buscando. Si es menor, vuelve a contar los 4 saltos, si es mayor, tiene que ir preguntando hacia atrás, niño por niño, si es el valor que busca. 
Una vez encuentra el valor, se da por finalizada la búsqueda.

\section{Pasos a seguir}
Antes de desarrollar el juego, hay que preparar las cartulinas. En cada una de ellas se escribirá previamente un número del 1 al 20 (por ejemplo), y cada uno de los niños escogerá una de ellas. 
Luego, se colocorán en orden de menor a mayor de acuerdo al número que tienen en sus láminas. 
Cuando el niño que busca les pregunta por su valor, deben revelar el número que tienen en la cartulina. 
Al juego finaliza cuando se encuentra el valor. 

\end{document}
